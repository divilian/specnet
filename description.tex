\documentclass[12pt,handout]{beamer}
\usepackage{mathtools}
\useoutertheme{infolines}
\usetheme{default}
\usefonttheme{serif}
%\usefonttheme{structuresmallcapsserif}
\hypersetup{colorlinks=true,linkcolor=blue}
\setbeamertemplate{navigation symbols}{}
\author{Stephen Davies}
\definecolor{darkgreen}{rgb}{0,.5,0}
\usepackage{MnSymbol}
\newcommand{\freakingtilde}{\raisebox{0.5ex}{\texttildelow}}
\begin{document}

\begin{frame}[c]{What you're seeing}

\begin{itemize}
\pause
\item Geometric positioning ($x,y$ position of nodes) has \textbf{no meaning}.
\pause
\item There is no landscape at all. An agent's state has only these attributes:
    \begin{itemize}
    \item A wealth (floating-pt number $> 0$).
    \item Its associates in the social network (mostly static for now).
    \item Whether it's in a proto-institution, and if so, with whom.
    \end{itemize}

\pause
\item Wealth profit/loss is currently completely random (nothing to do with
``how rich your sugar square is,'' because there are no sugar squares).
\end{itemize}

\end{frame}

\begin{frame}[c]{The process}

\begin{enumerate}
\small
\item Agents begin on a random (connected) social network. They are each given
an initial wealth \freakingtilde U(0 to 100).

\pause
\item At each iteration:

    \begin{enumerate}
    \item Two agents are chosen at random:

        \begin{enumerate}
        \item The first is chosen at random.
        \item With probability $o$ (``openness''), the second node is chosen
from the graph at large. With probability $1-o$, the second node is chosen from
the first node's neighbors.
        \end{enumerate}

\pause
    \item If neither agent is in a proto-institution, and if both of their
wealths are greater than $T$ (threshold; currently 50.0), they form a
proto-institution. (In this case, if not already neighbors, they are made
neighbors.)

\pause
    \item All agents are given a ``wealth adjustment'' \freakingtilde
U(-.5,.5)

\pause
    \item Agents with $< 0$ wealth die.
    \end{enumerate}

\end{enumerate}

\end{frame}

\begin{frame}[c]{Visualization}

In the visualization, a node's \textbf{size} is related to its \textbf{wealth}.

\pause
\bigskip

A node's \textbf{color} shows its \textbf{proto-institution}. (White indicates
none. Black indicates the node has died.)

\end{frame}

\begin{frame}[c]{Not done yet}

\pause
\begin{itemize}
\itemsep.1em
\item Proto-institutions actually functioning:
    \begin{itemize}
    \item \textbf{Pooling} of resources (so members are prevented from dipping
below zero wealth.)
\pause
    \item \textbf{Investment} (\textit{i.e.}, membership in a proto-institution
can increase one's wealth.)
    \end{itemize}
\pause
\item \textbf{More than two} agents per proto-institution.
\pause
\item Proto-institutions \textbf{dissolving} when appropriate.
\pause
\item Agents \textbf{learning} about the concept of proto-institutions by being
exposed to them (even if not accepted).
\pause
\item (A dozen other things that Rajesh and Milton will now name.)
\end{itemize}

\end{frame}

\end{document}
